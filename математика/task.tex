\documentclass[a4paper,12pt]{article}

\usepackage[utf8]{inputenc}
\usepackage[russian]{babel}
\usepackage{xstring}
\usepackage{amsmath}
\usepackage{amsfonts}
\usepackage{amssymb}
\usepackage{amsthm}
\usepackage{enumitem}
\usepackage[svgnames]{xcolor}
\usepackage{sectsty}

\voffset=-25.4mm
\topmargin=25mm
\headheight=0mm
\headsep=0mm
\textheight=247mm

\hoffset=-25.4mm
\oddsidemargin=20mm
\evensidemargin=20mm
\textwidth=170mm

\allsectionsfont{\raggedright\rmfamily}
\makeatletter
\def\@seccntformat#1{\csname the#1\endcsname.~}
\makeatother



\makeatletter
\def\eatrelax #1#2EndOfControl{#1#2}\relax
\def\createverb#1{%
    \begingroup
    \xdef\bbb{\detokenize{#1}}
    \let\do\@makeother
    \dospecials
    \catcode`\@=12
    \catcode`\{=12
    \catcode`\}=12
    \catcode`\\=12
    \catcode`\ =10
    \tokenize{\verbtext}{\bbb}%
    \ttfamily\expandafter\eatrelax\verbtext EndOfControl
    \endgroup
}%
\def\hmrg{.7em}
\def\vmrg{.7ex}
\def\frm#1#2#3{\fcolorbox{black}{#3}{\hspace*{\hmrg}\begin{minipage}{#1}\vspace*{\vmrg}#2\vspace*{\vmrg}\end{minipage}\hspace*{\hmrg}}}
\def\tsk#1{\fcolorbox{black}{MistyRose}{\hspace*{\hmrg}\begin{minipage}{.8\textwidth}#1\end{minipage}\hspace*{\hmrg}}}
\def\illustr#1#2#3{\par\vspace*{\vmrg}\noindent\frm{.55\textwidth}{\createverb{#1}}{#2}\quad \frm{.35\textwidth}{#1}{#3}\par\vspace*{\vmrg}}
\def\noillustr#1#2#3{\par\vspace*{\vmrg}\noindent\frm{.55\textwidth}{\createverb{#1}}{#2}\par\vspace*{\vmrg}}
\def\illustrate#1{\illustr{#1}{Moccasin}{Azure}}
\def\noillustrate#1{\noillustr{#1}{Moccasin}{Azure}}
\def\nillustrate#1{\illustr{#1}{yellow}{Pink}}
\makeatother

\newtheorem{thr}{Теорема}
\newtheorem{crlx}[thr]{Следствие}
\newtheorem{lmm}[thr]{Лемма}
\newtheorem{clm}[thr]{Утверждение}
\newtheorem{dfn}{Определение}
\newtheorem{ax}{Аксиома}

\newtheoremstyle{claim}{1mm}{.2mm}{\slshape}{}{\scshape}{:}{.5em plus 3em}{}
\theoremstyle{claim}
\newtheorem{crl}[thr]{Следствие}

\newtheoremstyle{theor}{1mm}{.2mm}{\upshape\sffamily}{}{\scshape}{:}{.5em}{}
\theoremstyle{theor}
\newtheorem{thrx}[thr]{Теорема}

\title{Набор математических формул в \LaTeX}
\author{}
\date{}
\begin{document}
\maketitle
\large

\begin{itemize}
\item [1)] \tsk{Пусть $x,y,z$ — элементы множества $X\times Y$}
\item [6)] \tsk{Среди чисел $n! + 2, \ldots ,n! + n$ нет простых}
\item [11)] \tsk{Условие непрерывности функции $f$ $$ \forall \epsilon > 0 \: \exists \delta > 0 (|x-y|<\delta \to |f(x)-f(y)|<\varepsilon)$$}
\item [16)] \tsk{Нахождение количества сочетаний из $n$ по $m$: $$C^m_{n}=\frac{n(n-1)(n-2) \cdots (n-m+1)}{m!}=\frac{n!}{m!(n-m)!} $$}
\item [21)] \tsk{Из основного тригонометрического тожества $$\sin^2x+\cos^2x=1$$ делением на $\cos^2x$ можно получить равенство $$1+\tg^2x=\frac{1}{\cos^2x}$$}
\item [26)] \tsk{Градиент скалярной функции:$$\nabla f=\frac{\partial f}{\partial x}\vec i+\frac{\partial f}{\partial y}\vec j+\frac{\partial f}{\partial z}\vec k$$}
\item [31)] \tsk{Теорема Кёнига утверждает, что при $\alpha_{i}<\beta_{i}$ для всех $i \in I$ будет выполнено $$\sum _{i \in I}\alpha_i < \prod _{i \in I}\beta_i $$}
\item [36)]\tsk{Прямое декартово произведение групп $\mathfrak{A}_{i}$, $i \in I$, снова будет группой: $$\mathfrak{B}=\prod _{i \in I}\mathfrak{A}_{i}$$}
\item [41)] \tsk{Якобиан преобразования $\vec y = \vec f(\vec x)$ находится как определитель 
$$\left | \begin {matrix} 
	\frac{\partial f_{1}}{\partial x_{1}} & \frac{\partial f_{1}}{\partial x_{2}} \\ 
	\frac{\partial f_{2}}{\partial x_{1}} & \frac{\partial f_{2}}{\partial x_{2}} \\ 
\end {matrix} \right | 
$$}
\end{itemize}


\end{document}

